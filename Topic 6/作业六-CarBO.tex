\documentclass[utf8]{ctexart}
\usepackage{mathtools}
\usepackage{amsmath}
\title{\zihao{1}作业六}
\author{\zihao{-3}CarBO}
\date{}
\begin{document}
\maketitle
    \section{单位四元数$q$可以表示旋转。
            一个三维空间点可以用虚四元数$p$表示,
            用四元数$q$旋转点$p$的结果为$p^\prime$的结果为\[p^\prime=qpq^{-1}\]
            证明:此时$p^\prime$必定为虚四元数}
        因为$q$为单位四元数,因此$q^{-1}=q^*$。
        设$p=[0 \ v_p],q=[s\ v]$,则
        \[\begin{aligned}
            p^\prime &=qpq^{-1}=qpq^*\\
                    &=[s\ v][0 \ v_p][s\ -v]\\
                    &=[-v^Tv_p \quad sv_p+v\times v_p][s\ -v]
        \end{aligned}\]
        对于上式的实部来说,有
        \[\begin{aligned}
            &-sv^Tv_p+(sv_p+v\times v_p)^Tv\\
            =&-sv^Tv_p+sv_p^Tv+(v\times v_p)^Tv\\
            =&0
        \end{aligned}\]
        证毕。
\end{document}